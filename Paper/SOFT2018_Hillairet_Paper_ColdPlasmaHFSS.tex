\documentclass[review]{elsarticle}

\usepackage{lineno,hyperref}
\modulolinenumbers[5]

\journal{Journal of \LaTeX\ Templates}

%%%%%%%%%%%%%%%%%%%%%%%
%% Elsevier bibliography styles
%%%%%%%%%%%%%%%%%%%%%%%
%% To change the style, put a % in front of the second line of the current style and
%% remove the % from the second line of the style you would like to use.
%%%%%%%%%%%%%%%%%%%%%%%

%% Numbered
%\bibliographystyle{model1-num-names}

%% Numbered without titles
%\bibliographystyle{model1a-num-names}

%% Harvard
%\bibliographystyle{model2-names.bst}\biboptions{authoryear}

%% Vancouver numbered
%\usepackage{numcompress}\bibliographystyle{model3-num-names}

%% Vancouver name/year
%\usepackage{numcompress}\bibliographystyle{model4-names}\biboptions{authoryear}

%% APA style
%\bibliographystyle{model5-names}\biboptions{authoryear}

%% AMA style
%\usepackage{numcompress}\bibliographystyle{model6-num-names}

%% `Elsevier LaTeX' style
\bibliographystyle{elsarticle-num}
%%%%%%%%%%%%%%%%%%%%%%%

\begin{document}

\begin{frontmatter}

\title{Ion Cyclotron Frequency Range Cold Magnetized Plasma Modelling in {ANSYS} {HFSS}}
%% Group authors per affiliation:
\author[IRFM]{J.Hillairet\corref{mycorrespondingauthor}}
\author[ERM]{R.Ragona}

\cortext[mycorrespondingauthor]{Corresponding author}
\ead{julien.hillairet@cea.fr}

\address[IRFM]{CEA, IRFM, F-13108 Saint Paul-lez-Durance, France}
\address[ERM]{Laboratory for Plasma Physics, Royal Military Academy, (LPP-ERM/KMS), BE-1000, Brussels, Belgium}


\begin{abstract}
Abstract here
\end{abstract}

\begin{keyword}
ICRH \sep Cold Plasma \sep Finite Elements
%\MSC[2010] 00-01\sep  99-00
\end{keyword}

\end{frontmatter}

\linenumbers

\section{Introduction}


Explicit the needs and objectives or the work
Be clear on what we not intend to do : hot plasma, core/edge physics, sheaths physics

Modeling the plasma as a dielectric: recipe ? Can an increasing permittivity dielectric could compare  coupling on an increasing density edge plasma? 

Gyrotropic plasma : sigma trick to deal with boundaries.
Loss : recipe to define the sigma increase ? Does it depends of the power ?
45$\deg$ plasma with Master/Slave BC. 



How do these results extrapolate to largest structure ? Impact of n// vs sigma and size of plasma volume? 

MFEM FE plasma \cite{Shiraiwa2017}
Salted water as dielectric \cite{Ravera2012, Bottollier-Curtet2011}
Dielectric as plasma \cite{Messiaen2011}
PML \cite{Becache2016}

\section*{References}

\bibliography{SOFT2018}

\end{document}