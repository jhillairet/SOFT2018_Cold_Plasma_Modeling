\documentclass[preprint,3p,twocolumn]{elsarticle}

\usepackage{lineno}
\usepackage[draft]{hyperref}
%\modulolinenumbers[5]

\journal{Fusion Engineering and Design}
\bibliographystyle{elsarticle-num}
%%%%%%%%%%%%%%%%%%%%%%%

\begin{document}

\begin{frontmatter}

\title{Ion Cyclotron and Lower Hybrid Frequency Range Cold Magnetized Plasma Modelling in {ANSYS} {HFSS}}
%% Group authors per affiliation:
\author[IRFM]{Julien Hillairet\corref{mycorrespondingauthor}}
\author[ERM]{Riccardo Ragona}
\author[IRFM]{Laurent Colas}
\author[IRFM]{Walid Helou}
\author[ANSYS]{Frédéric Bocquet}

\cortext[mycorrespondingauthor]{Corresponding author}
\ead{julien.hillairet@cea.fr}
%% Authors affiliations
\address[IRFM]{CEA, IRFM, F-13108 Saint Paul-lez-Durance, France}
\address[ERM]{Laboratory for Plasma Physics, Royal Military Academy, (LPP-ERM/KMS), BE-1000, Brussels, Belgium}
\address[ANSYS]{ANSYS France}

\begin{abstract}
The modelling of the interactions between cold magnetized plasmas and Ion Cyclotron and Lower Hybrid Resonance Frequency antennas is generally assessed using specifically developed codes. These antenna coupling codes often approximate the plasma to surface impedances described by 1D half infinite models. Depending of the mathematical approaches used by these codes, the modelled antennas can be described either in simplified 2D dimensions or in 3D complex geometries converted from CAD models. Such approaches add an additional step of model approximation/conversion which does not allow one to work directly on realistic antenna model and to assess the impact of changes. Moreover, during the design phase, it is convenient to use the simulation to directly estimate the thermal and mechanical loads in the same software suite. 


\end{abstract}

\begin{keyword}
ICRH \sep Cold Plasma \sep Finite Elements
%\MSC[2010] 00-01\sep  99-00
\end{keyword}

\end{frontmatter}

\linenumbers

\section{Introduction}
Achieving the plasma temperature expected for nuclear fusion requires external heating systems, such as dedicated Radio-Frequency antennas. Dimensions, power level and manufacturing cost which are at stake make it impossible to build scale-one mock-up during design and prototyping phases. For that reason, modelling the electromagnetic interactions between magnetized plasmas and Radio-Frequency antennas is mandatory for nuclear fusion research. 

During the last two decades, the availability of RF full wave software eased the design of Ion Cyclotron ({ICRF}) and Lower Hybrid Resonance Frequency systems ({LHRF}). As the both software and hardware progressed, the modelling of always more realistic and bigger components became possible, reducing the gap between CAD and RF models and accelerating the necessary feedback between mechanical and RF engineers. Nowadays, thermal or mechanical loads can be provided directly from the results of RF simulations inside integrated workflows. However, antenna to plasma coupling modelling with such tools was not possible unless severe simplifications. Indeed, full wave codes didn't support since recently anisotropic and non homogeneous dielectric media and this is the reason why community codes such as {ANTITER II} \cite{Messiaen2011} or {TOPICA} \cite{Lancellotti2006} for {ICRF} and {TOPLHA} \cite{Milanesio2012}, {OLGA} \cite{Preinhaelter2017} or {ALOHA} \cite{Hillairet2010a} for LHRF, were developed specifically for that purpose. While these tools are generally faster than full-wave modelling, they mostly assume slab plasma and simplified antenna geometries. % dealing to deal with plasma poloidal and toroidal curvatures.  


Regarding to coupling performances only, using constant dielectric for the antenna loading is a good workaround to approximate the ICRF plasma coupling \cite{Messiaen2011a}. 
%(but not LH, why?). 
% test on water load : \cite{Messiaen2005}+Helou
As a recent illustration, antenna coupling performances have been investigated substituting the plasma load  with a salty water tank with ANSYS HFSS \cite{Ravera2012} or with BaTiO3 with Microwave CST \cite{Bottollier-Curtet2011}. % OTHER REF ?



Since the last decade, full wave codes such as {COMSOL}, Microwave {CST} or {ANSYS} {HFSS} for example, are able to define propagating medium as anisotropic tensor and eventually inhomogeneous and frequency dependent. 
This ability allowed coupling\cite{Meneghini2009g, Shiraiwa2009} and absorption\cite{Meneghini2009} calculations of the C-Mod LH launcher. Recently, the open-source initiative \cite{Shiraiwa2017} allows modelling RF waves propagation in SOL plasmas. However, in all cases, a difficulty arises in the model setup concerning the boundary conditions at the edges of the cold plasma domains. Indeed, default radiation or absorbing boundary conditions such as Perfectly Matched Layer (PML) available is all these software do not take into account the case of a \cite{Jacquot2013b}
Depending the geometry considered, workarounds consist in adding artificial losses in order to attenuate the waves before they reach the propagation domain edges\cite{bibid} or to use symmetry boundary conditions to model axis-symmetric geometries.   


  


demonstration of the full-wave calculation of the propagation and the absorption of LHRF waves \cite{Meneghini2009, } 




In \cite{Louche2017}, a 1D cold plasma has been modelled in CST Microwave Studio against analytical solutions. Due to a limitation in CST to defined non-homogeneous medium, a stratified PML has been used to mimic the single pass absorption in a bulk plasma. To obtain a precise solution, a large number of layers are required in the PML are needed which is cumbersome to define (is it?).   

This allowed first LHRF coupling calculation 



Thanks to the collaboration between the fusion community and ANSYS, the RF modelling software ANSYS HFSS supports non-homogeneous gyrotropic medium, which are used to describe magnetized cold plasma away from resonances. 

In this paper, ANSYS HFSS is used to model ICRH antennas coupling to cold magnetized plasma. 

A simplified antenna model is compared with the specific coupling code ANTITER for various plasma density profiles. It is found that the coupling performances can generally be reproduced in HFSS. Its finite-element implementation imposes inherent restrictions on the definition and boundaries of the plasma domain due to the gyrotropic media in which two propagation modes can co-exist. 

These restrictions are discussed and technical recipes are given to conduct satisfying antenna coupling calculations with ANSYS HFSS, which allow faster design phases and experimental comparisons. 

Be clear on what we not intend to do : hot plasma, core/edge physics, sheaths physics

Modeling the plasma as a dielectric: recipe ? Can an increasing permittivity dielectric could compare  coupling on an increasing density edge plasma? 

Gyrotropic plasma : sigma trick to deal with boundaries.
Loss : recipe to define the sigma increase ? Does it depends of the power ?
45$\deg$ plasma with Master/Slave BC. 



How do these results extrapolate to largest structure ? Impact of n// vs sigma and size of plasma volume? 

MFEM FE plasma \cite{Shiraiwa2017}

Dielectric as plasma
PML \cite{Becache2016}




\section*{References}

\bibliography{SOFT2018}

\end{document}